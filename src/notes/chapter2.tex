%! Author = Len Washington III
%! Date = 1/17/24

% Preamble
\documentclass[title={Chapter 2}]{fdsn201notes}

% Packages

% Document
\begin{document}

\maketitle
\setcounter{chapter}{2}

\section{What Is a Healthful Diet?}\label{sec:what-is-a-healthful-diet?}
\begin{itemize}
	\item A healthful diet is
	\begin{itemize}
		\item Adequate
		\item Moderate
		\item Balanced
		\item Nutrient-dense
		\item Varied
	\end{itemize}
\end{itemize}

\section{A Healthful Diet is$\dots$}\label{sec:a-healthful-diet-is$dots$}
\subsection{Adequate}\label{subsec:a-healthful-diet-is-adequate}
\begin{itemize}
	\item An adequate diet provides enough energy, nutrients, and fiber to support a person's health
	\item A diet adequate in one area can still be inadequate in another
	\item A diet adequate for one person may not be adequate for another
\end{itemize}

\subsection{Moderate}\label{subsec:a-healthful-diet-is-moderate}
\begin{itemize}
	\item Another key to a healthful diet is moderation
	\item A healthful diet contains the right amounts of foods for maintaining proper weight and nutrition
\end{itemize}

\subsection{Nutrient-Dense}\label{subsec:a-healthful-diet-is-nutrient-dense}
\begin{itemize}
	\item A nutrient dense diet is made up of foods and beverages that supply the highest level of nutrients for the lowest number of calories
	\item Examples of nutrient dense foods are fruits, vegetables and whole grain
\end{itemize}

\subsection{Balanced}\label{subsec:a-healthful-diet-is-balanced}
\begin{itemize}
	\item A balanced diet contains the right combinations of foods to provide the proper proportions of nutrients
\end{itemize}

\subsection{Varied}\label{subsec:a-healthful-diet-is-varied}
\begin{itemize}
	\item \definition{Variety}{eating many different foods from the different food groups on a regular basis}
	\item A healthful diet is not based on only one or a few types of foods
\end{itemize}

\section{What's Behind Our Food Choices?}\label{sec:what's-behind-our-food-choices?}
\begin{itemize}
	\item Hunger is a basic biological urge, while appetite is a psychological desire influenced by
	\begin{itemize}
		\item Sensory data
		\begin{itemize}
			\item Social and cultural cues
			\item Sight
			\item Smell
			\item Taste
			\item Texture
			\item Sound
		\end{itemize}
		\item Social, cultural, and emotional cues
		\begin{itemize}
			\item Craving ``comfort foods''
			\item Associating food with a location
			\begin{itemize}
				\item Popcorn at the movies, or hot dogs at a baseball game
			\end{itemize}
		\end{itemize}
		\item Learned factors (family, community, religion)
		\begin{itemize}
			\item Conditioned taste aversion: avoidance of a food as a result of a negative experience such as an illness
		\end{itemize}
	\end{itemize}
\end{itemize}

\section{Designing a Healthful Diet}\label{sec:designing-a-healthful-diet}
\begin{itemize}
	\item Tools for designing a healthful diet include:
	\begin{itemize}
		\item Food labels
		\item 2010 Dietary Guidelines for Americans
		\item The USDA Food Patterns and MyPlate graphic
		\item Other eating plans
	\end{itemize}
\end{itemize}

\section{Food Labels}\label{sec:food-labels}
\begin{itemize}
	\item Five components of food labels:
	\begin{itemize}
		\item Statement of identity
		\item Net contents of the package
		\item Ingredient list
		\item Name and address of the food manufacturer, packer, or distributor
		\item Nutrition information
	\end{itemize}
\end{itemize}

\subsection{Nutrition Facts Panel}\label{subsec:nutrition-facts-panel}
\begin{itemize}
	\item The Nutrition Facts Panel contains the nutrition information required by the FDA
	\begin{itemize}
		\item Label regulations began in 1973
		\item The U.S.\ Food and Drug Administration (FDA) has made changes to the 20-year old nutrition labels on packaged foods.
		The changes to the nutrition label provide information to help compare products and make healthy food choices.
	\end{itemize}
	\item This information can be used in planning a healthful diet
	\item Serving size and servings per container
	\begin{itemize}
		\item Serving sizes can be used to plan appropriate amounts of food
		\item Standardized serving sizes allow for comparisons among similar products
	\end{itemize}
	\item Calories and Calories from fat per serving
	\begin{itemize}
		\item This information can be used to determine if a product is relatively high in fat
	\end{itemize}
	\item List of nutrients
	\begin{itemize}
		\item Fat (total, saturated, and \emph{trans})
		\item Cholesterol
		\item Sodium
		\item Carbohydrates
		\item Protein
		\item Some vitamins and minerals
	\end{itemize}
	\item Percent Daily Values (\%DV)
	\begin{itemize}
		\item Describe how much a serving of food contributes to your total intake of a nutrient
		\item Based on a diet of 2,000 Calories per day
		\item Can be used to determine if a product is low or high in a particular nutrient
		\item Based on:
		\begin{itemize}
			\item Reference Daily Intakes (RDIs)\label{dfn:rdi} for foods with a Recommended Dietary Allowance (RDA)\label{dfn:rda} value
			\item Daily Reference Values (DRVs)\label{dfn:drv} for foods without an RDA value
		\end{itemize}
	\end{itemize}
	\item Footnote
	\begin{itemize}
		\item Contains general dietary advice for all people of all health
		\item Must be present on all food labels
		\item Also compares a 2,000-Calorie diet with a 2,500-Calorie diet
	\end{itemize}
\end{itemize}

\subsection{Nutrient Claims on Food Labels}\label{subsec:nutrient-claims-on-food-labels}
\begin{itemize}
	\item The FDA has approved several claims related to health and disease
	\item If current scientific evidence about a health claim is not convincing, the label may have to include a disclaimer
	\item \definition{Structure}{function claims such as ``Builds stronger bones'' can be made with no proof and therefore no actual benefits may be seen}
\end{itemize}

\begin{table}[H]
	\centering
	\begin{threeparttable}
		\caption{FDA-Approved Terms and Definitions}
		\label{tab:fda-approved-terms-and-definitions}
%		\rowcolors{2}{rowmedgreen}{rowlightgreen}
		\begin{tabular}{p{0.11\textwidth} p{0.3\textwidth} p{0.6\textwidth}}
			\rowcolor{rowdarkgreen}\textbf{Nutrient} & \textbf{Claim} & \textbf{Meaning}\\
			\rowcolor{rowlightgreen}Energy & Calorie free

			Low Calorie

			Reduced Calorie & Less than 5 kcal per serving

			40 kcal or less per serving

			At least 25\% fewer kcal than reference (or regular) food\\
			\rowcolor{rowmedgreen} & Fat free & Less than 0.5 g of fat per serving\\
			\rowcolor{rowmedgreen} & Low fat & 3 g or less fat per serving\\
			\rowcolor{rowmedgreen} & Reduced fat & At least 25\% less fat per serving than reference food\\
			\rowcolor{rowmedgreen} & Saturated fat free & Less than 0.5 g of saturated fat \emph{and} less than 0.5 g of \emph{trans} fat per serving\\
			\rowcolor{rowmedgreen} & Low saturated fat & 1 g or less saturated fat and less than 0.5 g \emph{trans} fat per serving \emph{and} 15\% or less of total kcal from saturated fat\\
			\rowcolor{rowmedgreen} & Reduced saturated fat & At least 25\% less saturated fat \emph{and} reduced by more than 1 g saturated fat per serving as compared to reference food\\
			\rowcolor{rowmedgreen} & Cholesterol free & Less than 2 mg of cholesterol per serving \emph{and} 2 g or less saturated fat and \emph{trans} fat combined per serving\\
			\rowcolor{rowmedgreen} & Low cholesterol & 20 mg or less cholesterol \emph{and} 2 g or less saturated fat per serving\\
			\rowcolor{rowmedgreen} \multirow[t]{-9}{0.11\textwidth}{Fat and Cholesterol} & Reduced cholesterol & At least 25\% less cholesterol than reference food \emph{and} 2 g or less saturated fat per serving\\
			\rowcolor{rowlightgreen} & High fiber & 5 g or more fiber per serving* \\
			\rowcolor{rowlightgreen} & Good source of fiber & 2.5 g to o4.9 g fiber per serving \\
			\rowcolor{rowlightgreen} & More or added fiber & At least 2.5 to 4.9 g fiber per serving \\
			\rowcolor{rowlightgreen} & Sugar free & Less than 0.5 g sugars per serving \\
			\rowcolor{rowlightgreen} & Low sugar & Not defined; no basis for recommended intake \\
			\rowcolor{rowlightgreen} & Reduced/less sugar & At least 25\% less sugars per serving than reference food \\
			\rowcolor{rowlightgreen} \multirow[t]{-7}{0.11\textwidth}{Fiber and Sugar} & No added sugars or without added sugars & No sugar or sugar-containing ingredient added during processing \\
		\end{tabular}
		\begin{tablenotes}
			\small
			\item *High-fiber claims must also meet the definition of low fat; if not, then the level of total fat must appear next to the high-fiber claim.
			Data adapted from: ``Food Labeling Guide'' (U.S.\ Food and Drug Administration)
		\end{tablenotes}
	\end{threeparttable}
\end{table}

\end{document}
