%! Author = Len Washington III
%! Date = 1/8/24
%! compiler = pdflatex

% Preamble
\documentclass[title={Credible Sources - Formulate Question},points={20}]{fdsn201homework}

% Document
\begin{document}

\maketitle

\noindent%
\textbf{Description}: In Part 1 of this assignment, you will identify and fine-tune your ``burning question'' into one that is specific enough to lead you to a credible, peer-reviewed journal publication.
Review ``Credible Sources Assignment: Overview'' document for full details and grading rubric.

\begin{problems}
	\item \textbf{5 points. Please state your ``burning question''.}
	Start by asking yourself$\dots$ ``What about nutrition, health and/or wellness have I always wanted to know more, but am unsure how to get started?''
	This is your ``burning question''.
	\begin{answer}
		My burning question is ``how can I stop myself from indulging my food cravings, or stop cravings all together?''
	\end{answer}
	\item \textbf{5 points. Why is this question important for you to answer?}
	Elaborate on why you selected this question.
	\begin{answer}
		This is important to me because since I was a kid, I've found it very difficult to resist food cravings, especially when I was in an environment where I had easy access to junk food.
		As I've gotten older, the only way I could resist cravings was if I didn't have anything at home that wasn't good for me, which somewhat worked.
		This hasn't been fool-proof though, because there were always vending machines or 7-11's close enough that if I had a craving, it would not be difficult to get what I wanted.

		This is the primary reason for why I gain a lot of weight. During the times when I cut out late night convenience store runs, I lost about 15 pounds in a week from the reduced calories and no other big changes.
		As soon as I started going back to the store, I gained it all back. I want to be able to cut out those cravings, or at least control them enough to where I'm not constantly eating, and I can actually keep off empty calories and pounds.
	\end{answer}
	\item \textbf{5 points. How can you make your question more specific?
	Please describe your thought process.}
	Visit the \href{https://guides.lib.byu.edu/c.php?g=216340&p=1428396}{BYU Library Finding and Narrowing Your Topic} page for tips in developing shaping your question.
	State your question below.
	\begin{answer}
		I can make this question more specific by researching studies where similar results were attained by changing a person's diet or some psychological trick to stop them from wanting to eat that specific junk food.%
		\footnote{One specific psychological example I initially thought of was eating the foods when I'm extremely sick, so I can create an aversion to it, which I inadvertently did recently with some flavors of chips.
		I know it's not healthy and I wouldn't intentionally do that to myself, but it would have similar results to what I was thinking of.}
	\end{answer}
	\item \textbf{5 points. Please restate your ``burning question'' in the most specific, concise wording possible.}
	\begin{answer}
		My reformed burning question is ``What ways can a person reduce or control their cravings to reduce the amount of calories they consume?''
	\end{answer}
\end{problems}

\end{document}