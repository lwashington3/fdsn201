%! Author = Len Washington III
%! Date = 1/8/24

% Preamble
\documentclass[title={Credible Sources - Literature Review and Article Selection},points={30}]{fdsn201homework}
\addbibresource{cred-sources.bib}
\let\oldhref\href
\renewcommand{\href}[2]{\oldhref{#1}{\underline{#2}}}

% Document
\begin{document}

\maketitle

\noindent%
\textbf{Description}: In Part 2 of this assignment, you will use the ``burning question'' you formed in Part 1 to identify key search words or terms, enter these key terms in a credible search engine to explore peer-reviewed journal publications, and finally, select a peer-reviewed journal publication that best answers your ``burning question''.
Review ``Credible Sources Assignment: Overview'' document for full details and grading rubric.

\begin{problems}
	\item \textbf{2 points. Restate your ``burning question'' in the most specific, concise wording possible.}
	Incorporate suggestions proved in class and/or from Part 1 feedback when restating your ``burning question''.%
	\begin{answer}%
		My burning question is ``What psychological options are available to help a person reduce their cravings and calorie intake?''
	\end{answer}
	\item \textbf{2 points. What changes have you made to your ``burning question'' based course readings, discussions, or other feedback and why?}
	If no changed have been made, please indicate how you used course content or prior experience to develop your ``burning question''.%
	\begin{answer}%
		The only change I made to my ``burning question'' was making the focus on reducing cravings to be psychological, since most of my cravings are due to an appetite that I'm mistaking for genuine hunger.
	\end{answer}
	\item \textbf{5 points. Provide a bulleted list of at least 5 key search words or terms (0.5 point per term), AND indicate why you selected each term (0.5 point per rationale).}
	Identify keywords or search terms that you plan to use when searching for peer-reviewed publications.
	Suggestions for how to identify and use search terms can be found using the \href{https://iit.libanswers.com/faq/292350}{IIT Galvin Library - Keywords (How do I find articles in databases?)}%
	\begin{answer}%
		\begin{itemize}
			\item psychological cravings -- If I can find information about the psychology behind cravings, then I might be able to find a solution to reducing those cravings.
			\item reduce cravings -- Similar logic as above, there may be research ways (even placebos) that can have been shown to help people reduce cravings
			\item reduce addictions -- I know that there have been studies that have claimed that in some cases, sugar could be as addicting as other substances, so learning how people have dealt with those addictions may help with sugar addictions and calorie intake.
			\item
			\item
		\end{itemize}
	\end{answer}
	\item \textbf{1 point. How many publications resulted when using your key words/terms to search for peer-reviewed publications using a credible search engine?}\\
	Visit the \href{https://library.iit.edu/}{IIT Galvin Library Search Page} use (IIT login information when prompted) to begin a search using the keywords/terms identified above.
	Narrow your search to ``articles'' published in peer reviewed \underline{journals} (please use journal articles only, not book chapters or other file types).%
	\begin{answer}%
		\href{https://i-share-iit.primo.exlibrisgroup.com/discovery/search?query=any,contains,psychology,AND&query=any,contains,reduce,AND&query=any,contains,cravings,AND&query=any,contains,calorie,AND&tab=CentralIndex&search_scope=CentralIndex&sortby=rank&vid=01CARLI_IIT:CARLI_IIT&lang=en&mode=advanced&offset=0}{31}, including an article reducing cocaine cravings and another for alcohol regulation.
	\end{answer}
	\item \textbf{1 point. Select and upload the peer-reviewed journal article that best answers your question.}\\
	Download the \textbf{full-text PDF} of this article and \textbf{upload PDF copy} of this article when submitting Part 2 of this assignment in Blackboard.%
	\begin{answer}%
		\href{run:Imaginal_retraining_reduces_craving_for_high-calorie_food.pdf}{Open full text PDF.}
	\end{answer}
	\item \textbf{5 points. Please cite your reference for your article in APA7 format.}\\
	See \href{https://apastyle.apa.org/style-grammar-guidelines/references}{American Psychological Association (APA) formatting - Version 7} for instructions.%
	\begin{answer}%
		\nocite{MORITZ2023106431}%
		\printbibliography[heading=none]
	\end{answer}
	\item \textbf{2 points. When was the article published?
	Is it still relevant or are there more recent publications that better answer your question?}
	\begin{answer}%
		The article was published on March 1, 2023 and is the most recent publication that I could find that related to my question.
	\end{answer}
	\item \textbf{2 points. What journal was this article published in?
	What is the aim of the journal?
	What is it trying to accomplish?}%
	\begin{answer}%
		The article was published in the journal \href{https://www.sciencedirect.com/journal/appetite}{Appetite}, which specializes in psychological and physiological influences on the intake of foods.
		The purpose of the journal is to help readers have a more intimate knowledge with how their brains perceive and want food.
	\end{answer}
	\item \textbf{2 points. Who are the authors of this article?
	What are their credentials or qualifications?}
	\begin{answer}%
		The authors of this article are Steffen Moritz, Anja Simone Göritz, Simone Kühn, Jürgen Gallinat and Josefine Gehlenborg.
		All of the authors except Göritz work in the Department of Psychiatry and Psychotherapy at the University Medical Center Hamburg, where Moritz is the current \href{https://orcid.org/0000-0001-8601-0143}{head of neuropsychology}.
		Göritz works at Behavioral Health Technology through the University of Augsburg, as well as the Lise Meitner Group for Environmental Neuroscience through the Max Planck Institute for Human Development.
	\end{answer}
	\item \textbf{2 points. Are these authors credible?
	Why or why not?
	Provide examples of things that could impact the credibility of an author.}
	\begin{answer}%
		These authors seem credible, each of them have links to their ORCID profiles which show tens of other peer-reviewed articles.
		Someting that could impact the credibility of an author is if they have been involved in papers with tampered data, or cannot provide any kind of data or process for a paper they recently wrote.
	\end{answer}
	\item \textbf{2 points. What affiliations, funding sources, conflicts of interest, or disclosures are mentioned?
	How might these factors impact credibility?}
	\begin{answer}%
		The affiliations mentioned are:
		\begin{enumerate}[label=\alph*]
		    \item University Medical Center Hamburg-Eppendorf, Department of Psychiatry and Psychotherapy, Martinistr. 52, D-20246, Hamburg, Germany
			\item Behavioral Health Technology, University of Augsburg, Augsburg, Germany
			\item Occupational and Consumer Psychology, Freiburg University, Engelbergerstraße 41, D-79085, Freiburg, Germany
			\item Lise Meitner Group for Environmental Neuroscience, Max Planck Institute for Human Development, Berlin, Germany
		\end{enumerate}
		There were no mentioned funding sources, conflicts of interest, or other means of disclosures mentioned.
		Given the fact that each author is affiliated with an accredited German university, I cannot say that there are any factors that would impact their credibility.
	\end{answer}
	\item \textbf{2 points. In your own words, summarize the purpose or objective of this article.
	Is it a review of works by others or reporting the results of a single study?}
	\begin{answer}%
		The purpose of this article is to determine the best imaginal retraining method to help reduce cravings in high-calorie food.
		It was stated that prior studies were limited, and that another purpose of the experiment was to have a much more diverse sample size than the few previous samples, which only had women as participants.
		The results of the study show that IR\footnote{Imagining yourself throwing your craved food at a wall or away from yourself.},
		while sometimes performing the motion had some evidence of being effective against cravings.
	\end{answer}
	\item \textbf{2 points. Based on all the information you found, is your article trustworthy?}\\
	\emph{Visit the \href{https://guides.lib.byu.edu/c.php?g=216340&p=1428399}{BYU Evaluating Credibility} page for tips.
	Explain your decision.}
	\begin{answer}%
		Based on all the information I found, I think that the article is trustworthy.
	\end{answer}
\end{problems}

\let\href\oldhref
\end{document}