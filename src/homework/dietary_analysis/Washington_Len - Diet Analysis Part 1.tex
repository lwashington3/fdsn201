%! Author = Len Washington III
%! Date = 3/23/24

% Preamble
\documentclass[
	title={Dietary Analysis Assignment - Nutrient Requirements},
	points={30}
]{fdsn201homework}

% Packages
\usepackage{soul}
\definecolor{tablegray}{HTML}{d9d9d9}
\newcommand{\header}[1]{\rowcolor{tablegray}\textbf{#1}}

% Document
\begin{document}

\maketitle

\chapter{Nutrient Requirements (30 points)}
\textit{Learning Objective -- Part 1:} To determine your personal nutrient requirements using credible sources to better understand your individual needs for key macro and micronutrients.\\\\
%
\textit{Description:} Learners will calculate daily nutrient requirements using credible sources and references.
Responses from Part 1 of this assignment will be used when completing Parts 2 \& 3.\\\\
%
\ul{Instructions -- Part 1:}
\begin{enumerate}[label=\arabic*.]
	\item Determine your individual requirements for key nutrients using the
	\href{https://www.dietaryguidelines.gov/sites/default/files/2020-12/Dietary_Guidelines_for_Americans_2020-2025.pdf}{2020-2025 Dietary Guidelines for Americans (DGA)}
	from
	\href{https://www.myplate.gov/}{MyPlate.gov},
	\href{https://ods.od.nih.gov/HealthInformation/nutrientrecommendations.aspx#:~:text=Recommended%20Dietary%20Allowance%20(RDA)%3A,assumed%20to%20ensure%20nutritional%20adequacy.}{National Institutes of Health Dietary Reference Intakes (NIH DRIs)},
	\href{https://www.myfitnesspal.com/}{myfitnesspal}, as well as those listed in the course text (Thompson \& Moore, 2018), or provided by instructor below.
	\item Answer the following questions to determine your requirements.
	Show your work and be sure to answer all questions.
	\item \textbf{Upload the following file:}:

	Part 1 assignment document as either a Word .docx or PDF file only
\end{enumerate}

\section{ENERGY REQUIREMENTS}\label{sec:energy-requirements}
\begin{enumerate}
	\item \textbf{Determine your resting energy requirements (REE) using the Harris-Benedict Equation, then calculate your total daily energy expenditure (TDEE) by adding an Activity Factor to the REE using the following information – show math:}
	\begin{equation*}
	\begin{aligned}
			&\mbox{MALE REE (kcal/day)} = 66.47 + [13.75 \times \mbox{weight (kg)}] + [5 \times \mbox{height (cm)}] – [6.77 \times \mbox{age (years)}] \\
			&\mbox{FEMALE REE (kcal/day)} = 655.1 + [9.56 \times \mbox{weight (kg)}] + [1.85 \times \mbox{height (cm)}] – [4.67 \times \mbox{age (years)}] \\
			&\mbox{TDEE (kcal/day)} = \mbox{REE (calculated above)} \times \mbox{activity factor}
	\end{aligned}
	\end{equation*}
	Activity Factors:
	\begin{description}
		\item[Sedentary:] Activity Factor 1.2 (little to no exercise, such as a desk job without additional physical activity)
		\item[Lightly Active:] Activity Factor 1.375 (light exercise 1--2 days/week)
		\item[Moderately Active:] Activity Factor 1.55 (moderate exercise 3--5 days/week)
		\item[Very Active:] Activity Factor 1.725 (hard exercise 6--7 days/week)
		\item[Extremely Active:] Activity Factor 1.9 (hard daily exercise, twice daily exercise, or physically demanding job)
	\end{description}
	\item What were the estimated daily energy requirements provided by \href{https://www.myfitnesspal.com/}{myfitnesspal}?
	How does it compare to your TDEE calculated using the Harris-Benedict Equation with activity factor above?

	\item What were the estimated daily energy requirements provided by your “My Plan” from MyPlate.gov?
	How does it compare to your TDEE calculated using the Harris-Benedict Equation with activity factor above?

	\item What happens to the excess energy (Calories) when you consume more than your body requires?

	\item What happens when you consume less energy (fewer Calories) than your body requires?
	How does your body obtain energy to meet your needs when intake is inadequate?
\end{enumerate}

\section{PROTEIN REQUIREMENTS}\label{sec:protein-requirements}
\begin{enumerate}[start=7]
	\item Calculate your protein requirements in grams per day using the acceptable macronutrient distribution range (AMDR).
	(hint – answer will be a range in grams/day based on your TDEE calculated using the Harris-Benedict Equation with activity factor above; use 4 kcal/gram protein to determine gram/day).
	Show math.
	\item Calculate your protein requirements in grams per day using the recommended daily allowance (RDA) – show math.
	\item What percent of your TDEE with activity factor does the RDA for protein make up?
	(Hint – there are 4 kcal/g protein) – show math.
	\item Does the low end of your AMDR range meet the RDA?
	How do your AMDR and RDA protein requirements compare?
	Which goal best meets your needs and why?
	\item What happens to the protein when you consume more than your body requires?
	How is it stored and/or eliminated?
	\item What happens when you consume less than your body requires?
	How does your body obtain amino acids to meet your needs when intake is inadequate?
\end{enumerate}

\section{CARBOHYDRATE REQUIREMENTS}\label{sec:carbohydrate-requirements}
\begin{enumerate}[start=13]
	\item Calculate your carbohydrate requirements in grams per day using the acceptable macronutrient distribution range (AMDR). (hint – answer will be a range in grams/day based on your TDEE calculated using the Harris-Benedict Equation with activity factor above; use 4 kcal/gram carbohydrate to determine gram/day).
	Show math.
	\item What is the recommended daily allowance (RDA) for carbohydrate for an adult and why?
	\item What percent of your TDEE with activity factor does the RDA for carbohydrate make up? (Hint – there are 4 kcal/g carb) – show math.
	\item Calculate your fiber requirements in grams per day using the adequate intake (AI) recommendations – show math.
	\item Calculate the upper limit for daily added sugar intake in grams per day using the Harris-Benedict Equation with activity factor above – show math.
	\item What happens to the carbohydrate when you consume more than your body requires?
	How is it stored and/or eliminated?
	\item What happens when you consume less than your body requires?
	How does your body obtain glucose to meet your needs when intake is inadequate?
\end{enumerate}

\section{FAT REQUIREMENTS}\label{sec:fat-requirements}
\begin{enumerate}[start=20]
	\item Calculate your fat requirements in grams per day using the acceptable macronutrient distribution range (AMDR). (hint – answer will be a range in grams/day based on your TDEE calculated using the Harris-Benedict Equation with activity factor above; use 9 kcal/gram carbohydrate to determine gram/day).
	Show math.
	\item Calculate your saturated fat intake range in grams per day based on your TDEE calculated using the Harris-Benedict Equation with activity factor above – show math.
	\item What are the recommendations for daily trans-fat intake?
	\item What happens to the fat when you consume more than your body requires?
	How is it stored and/or eliminated?
	\item What happens when you consume less than your body requires?
	How does your body obtain fat to meet your needs when intake is inadequate?
\end{enumerate}

\section{FLUID REQUIREMENTS}\label{sec:fluid-requirements}
\begin{enumerate}[start=25]
	\item What is the dietary reference intake (DRI) recommendation for your daily fluid requirements?
	\item What happens when you consume less fluid than your body requires?
	What are the symptoms and consequences of dehydration?
\end{enumerate}

\section{MICRONUTRIENT REQUIREMENTS – SODIUM \& POTASSIUM}\label{sec:micronutrient-requirements--sodium-&-potassium}
\begin{enumerate}[start=27]
	\item What is the adequate intake (AI) recommendation for your daily sodium requirements?
	\item What happens when you consume more sodium than your body requires?
	How can excess sodium intake contribute to hypertension (high blood pressure)?
	\item What is the adequate intake (AI) recommendations for your daily potassium requirements?
	\item What happens when you consume less potassium than your body requires?
	How can inadequate/low potassium intake contribute to hypertension (high blood pressure)?
\end{enumerate}

\end{document}